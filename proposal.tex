\documentclass{proposalnsf}
\usepackage{epsfig}
\usepackage{hyperref}
\usepackage{booktabs}
\usepackage{graphicx}
\usepackage{float}
\restylefloat{table}

% NSF proposal generation template style file.
% based on latex stylefiles  written by Stefan Llewellyn Smith and
% Sarah Gille, with contributions from other collaborators.

\newcommand{\jas}{{\it J. Atmos. Sci.}}
\newcommand{\jpo}{{\it J. Phys. Oceanogr.}}
\newcommand{\JPO}{{\it J. Phys. Oceanogr.}}
\newcommand{\jfm}{{\it J. Fluid Mech.}}
\newcommand{\jgr}{{\it J. Geophys. Res.}}
\newcommand{\JGR}{{\it J. Geophys. Res.}}
\newcommand{\jmr}{{\it J. Mar. Res.}}
\newcommand{\arfm}{{\it Ann. Rev. Fluid Mech.}}
\newcommand{\dsr}{{\it Deep-Sea Res.}}
\newcommand{\dao}{{\it Dyn. Atmos. Oceans}}
\newcommand{\jam}{{\it Journal of Applied Meteorology}}
\newcommand{\phfl}{{\it Phys. Fluids}}
\newcommand{\phfla}{{\it Phys. Fluids A}}
\newcommand{\PhilTrans}{{\it Philosophical Transactions of the Royal Society, 
London}}
\newcommand{\gafd}{{\it Geophys. Astrophys. Fluid Dyn.}}
\newcommand{\gfd}{{\it Geophys. Fluid Dyn.}}
\newcommand{\PCE}   {{\it Physics and Chemistry of the Earth}}
\newcommand{\PRL}   {{\it Physical Review Letters}}

\newcommand{\ProgOc}{{\it Prog. Oceanography}}
\newcommand{\WHOITR}{Woods Hole Oceanographic Institution Technical Report, WHOI-}
\newcommand{\degrees}{$\!\!$\char23$\!$}
%%% old lines below defined some mathematical fonts; these no longer seem necessary
%\DeclareFontFamily{OT1}{psyr}{}
%\DeclareFontShape{OT1}{psyr}{m}{n}{<-> psyr}{}
%\def\times{{\fontfamily{psyr}\selectfont\char180}}


\renewcommand{\refname}{\centerline{References cited}}

% this handles hanging indents for publications
\def\rrr#1\\{\par
\medskip\hbox{\vbox{\parindent=2em\hsize=6.12in
\hangindent=4em\hangafter=1#1}}}

\def\baselinestretch{1}

\begin{document}


\noindent{\Large{\bf Computational Modeling to Aid in Analysis and Interpretation of Multi-Modal Neutron Experiments }}\\*[3mm]

\pagenumbering{arabic}
\renewcommand{\thepage} {\arabic{page}}



\section*{Research Objectives}
The over-arching goal of our project is to create a streamlined workflow
for experimental data analysis and interpretation
needs of the neutron scattering community using atomistic modeling and
simulation. Neutron scattering experiments require users to model and
interpret data at the atomic level. With numerous software
applications and a large array of different file formats with each,
scientists tend to use a limited (and sometimes dated) subset of
software tools to tackle data analysis from neutron experiments. This
creates a barrier in their research between experimental and theoretical techniques.

We are currently developing a modeling and analysis workbench called the 
Integrated Computational Environment-Modeling \& Analysis for Neutrons,
or ICE-MAN. This is an extension of the Eclipse ICE project \cite{ICEwebsite}. We hope to create a seamless transition from both different
types of neutron scattering experimental data and computer modeling and
simulation techniques to tackle multi-modal data
analysis. An example of a workflow would be the study of a disordered
material. First, neutron scattering experiments could yield the average
structure of the material (diffraction data) and
the local structure via total scattering (the pair
distribution function, or PDF).  Molecular dynamics (MD) simulations can provide
a trajectory for an atomistic model  giving an
ensemble of possible atomic configurations to compare to experiment.
Sampled configurations from the trajectory could be used as inputs into reverse Monte Carlo modeling (RMC) to optimize the structures against the the diffraction and PDF data produced from the neutron scattering experiments.
The MD trajectories could also then be used to calculate the
inelastic neutron scattering spectrum of the system. This data could
feed back into the RMC modeling as a constraint. At present this process
would be exceedingly time consuming, involve expertise in multiple
techniques, and most likely present a barrier that would
seldom be overcome by general users.

We are requesting XSEDE's HPC resources to carry out this workflow on the outlined projects that already have a need for computational approaches to solve complex structural data analysis and also multiple potential projects for the general neutron scattering community, specifically Users of the Spallation Neutron Source (SNS) instruments at Oak Ridge National Lab. We propose three projects that have already been studied via neutron scattering experiments and have a need for atomistic modeling to aid in the data analysis process. These include irradiated amorphous silica for nuclear radiation damage on materials, high-pressure amorphous silica for cheaper manufacturing of electronics, and oxygen insertion into shafarzikite-like structures for oxygen storage materials.


\subsection*{Proposed Research}
\subsubsection*{Project 1: Investigate structural modifications of irradiated silica for nuclear materials}\label{lang}

The first project would be the study of the local structure changes in amorphous silica (SiO2) due to radiation damage. Previous studies have looked at the fine structure of ion tracks (narrow trails of permanent damage along penetrating heavy ion pathways) in thin film amorphous silica using small angle x-ray scattering (SAXS) measurements combine with non-equilibrium MD modeling and simulation techniques. These studies revealed that these ion tracks formed a shell around the path of penetration through the sample which consisted of a core lower in density and a shell high in density. The non-equlibrium MD calculations were carried out using an ~30k atom system where the ion track was produced by instantaneous deposition of kinetic energy to the atoms in the simulation cell. More recently, simulation sizes of ~600k atoms have also been carried out to compare to SAXS measurements as well. We intend to carry out similar simulations of comparable size that can be used to elucidate the structural changes observed in neutron scattering experiments of the average and local structure of these polymorphs of silica. Atomistic configurations from the end of this trajectory can then be fed into the RMC modeling to optimize the structure against the experimental data. The results of this study can help understand the fundamental degradation of silica materials exposed radiation damage and also to future work to manipulate nanoclusters within solid silica materials. 

\subsubsection*{Project 2: Modeling of amorphous germanium and silicon to determine high-pressure phase transformations}\label{haberl}

The second project we would like to undertake is trying to determine different structural changes of high-pressure amorphous germanium (a-Ge) and silicon (a-Si) from different forms of compression or indentation loads that result in new materials for industry applications. Similar to our first project, we will be modeling systems based on silicon that have been probed via neutron scattering studies to determine structure changes while under high-pressure, being compressed in diamond anvil cells or being indented using diamond tips. Under compression, these samples become metallic and do not return to their initial state upon releasing the pressure. Based on either indentation or compression, different structures are observed with improved electronic and photovoltaic properties with cheaper manufacturing methods. 

Our modeling and simulation effort will attempt to find atomistic models that best fit neutron scattering experiments probing the average and local atomic structure via diffraction and pair distribution functions along with vibrational density of states from vibrational spectroscopy. To create models of a-Ge and a-Si in agreement with all experimental data is still a current research challenge, and to date no such model exists. We have developed a structural relaxation program based on the Wooten-Winer-Weaire method to produce realistic random-network models of a-Ge and a-Si via the Atomic Simulation Environment (ASE). From this program, we have initial amorphous structures that retain their four-fold coordination and bond-angle deviations in qualitative agreement with experiment. We use these structures to then perform a range of equilibrium or non-equilibrium MD simulations to best replicate different pressure loading and thermal annealing cycles. We feed atomic configurations from the MD trajectories into reverse RMC modeling to optimize the structure against the experimental data taken throughout thermal and pressure load cycles and the send this back to the MD simulations for further refinement and progess in the cycle. Experimentally constrained methods have been pursued in recent studies with success using very similar Monte Carlo relaxation methods. These include pure Monte Carlo using just the experimental data and hybrid approaches that add an emperical potential as an additional constraint. Our hope that using ICE-MAN's unique capability of lowering the barrier for data cross-over from the MD simulations to the RMC optimization to fit to experimental data will allow a larger search in the phase space and accelerate the convergence of a best fit to a variety of experimental data with strong emphasis on neutron data for these materials.
\subsubsection*{Project 3: Modeling oxygen insertion in one-dimensional channels of shafarzikite-like structures}\label{deLaune}
Support work in understanding the atomic-level structural changes in shafarzikite-like (FeSb2O4) structures. Will look specifically at the oxygen insertion into cobalt and lead doped derivatives of shafarzikite. These materials show promise in applications of electro-catalysis due to their unique 1-D cation channels with high peroxide anion mobility and potential for high, directed electronic conductivity. Will help with correction and analysis of the neutron scattering data to push modeling efforts to clarify the oxidized structures.



Propose the research projects that we can answer.  \\
  3) Sankar's project \\





\section*{Computational Methodology (applications/codes)}
\subsection*{LAMMPS}
The main engine for the MD simulations will be the Large-Scale Atomistic/Molecular Massively Parallel Simulator (LAMMPS) software. This code is highly-scalable for a variety of machine architectures: IBM BG/L, Cray XT3, Cray XT5, and Intel clusters with GPUs (Comet). It can take advantage of GPU and Intel MIC accelerators with good performance scaling. For documentation on general benchmarking of LAMMPS, the following website contains several benchmark problems for a variety of machines: \href{http://lammps.sandia.gov/bench.html}{http://lammps.sandia.gov/bench.html}. 


\section*{Computational Research Plan}

This is the plan

\begin{table}[H]
  \caption{Summary of requested service units for projects}
  \resizebox{\textwidth}{!}{%
  \begin{tabular}{cccccc}
  	\toprule
  	Table & Project & Machine & Program & Service-Units & Storage (GB)\\
  	\midrule
  	1 & Irradiated SiO2   & Comet/Oasis & LAMMPS & 300,000 & 500 \\
  	2 & High-pressure a-Ge/a-Si & Comet/Oasis & LAMMPS & 300,000 & 500\\
  	3 & Oxygen in FeSb2O4 & Comet/Oasis & QE & 100,000 & 500\\
  	4 & RMC modeling   & OSG & RMCProfile & 25,000 & 100 \\
  	\bottomrule
  \end{tabular}
  }
\end{table}

Like it?

\section*{Justification for Service Units (SUs) Requested}

\section*{Additional Comments}


\bibliography{draft}
\bibliographystyle{jponew}


\end{document}

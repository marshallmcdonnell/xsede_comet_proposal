\documentclass{proposalnsf}
\usepackage{epsfig}

\usepackage{hyperref}
\usepackage{booktabs}
\usepackage{graphicx}
\usepackage{wrapfig}
\usepackage{float}
\usepackage[numbers,sort&compress,super]{natbib}
\usepackage[usenames, dvipsnames]{color}
\restylefloat{table}

% NSF proposal generation template style file.
% based on latex stylefiles  written by Stefan Llewellyn Smith and
% Sarah Gille, with contributions from other collaborators.

\newcommand{\jas}{{\it J. Atmos. Sci.}}
\newcommand{\jpo}{{\it J. Phys. Oceanogr.}}
\newcommand{\JPO}{{\it J. Phys. Oceanogr.}}
\newcommand{\jfm}{{\it J. Fluid Mech.}}
\newcommand{\jgr}{{\it J. Geophys. Res.}}
\newcommand{\JGR}{{\it J. Geophys. Res.}}
\newcommand{\jmr}{{\it J. Mar. Res.}}
\newcommand{\arfm}{{\it Ann. Rev. Fluid Mech.}}
\newcommand{\dsr}{{\it Deep-Sea Res.}}
\newcommand{\dao}{{\it Dyn. Atmos. Oceans}}
\newcommand{\jam}{{\it Journal of Applied Meteorology}}
\newcommand{\phfl}{{\it Phys. Fluids}}
\newcommand{\phfla}{{\it Phys. Fluids A}}
\newcommand{\PhilTrans}{{\it Philosophical Transactions of the Royal Society, 
London}}
\newcommand{\gafd}{{\it Geophys. Astrophys. Fluid Dyn.}}
\newcommand{\gfd}{{\it Geophys. Fluid Dyn.}}
\newcommand{\PCE}   {{\it Physics and Chemistry of the Earth}}
\newcommand{\PRL}   {{\it Physical Review Letters}}

\newcommand{\ProgOc}{{\it Prog. Oceanography}}
\newcommand{\WHOITR}{Woods Hole Oceanographic Institution Technical Report, WHOI-}
\newcommand{\degrees}{$\!\!$\char23$\!$}
%%% old lines below defined some mathematical fonts; these no longer seem necessary
%\DeclareFontFamily{OT1}{psyr}{}
%\DeclareFontShape{OT1}{psyr}{m}{n}{<-> psyr}{}
%\def\times{{\fontfamily{psyr}\selectfont\char180}}


\renewcommand{\refname}{\centerline{References cited}}

% this handles hanging indents for publications
\def\rrr#1\\{\par
\medskip\hbox{\vbox{\parindent=2em\hsize=6.12in
\hangindent=4em\hangafter=1#1}}}

\def\baselinestretch{1}

\begin{document}


\noindent{\Large{\bf Computational Modeling to Aid in Data Analysis and Interpretation of Neutron Experiments: From Irradiated Silica to Oxidization of Schafarzikite Structures  }}\\*[3mm]

\pagenumbering{arabic}
\renewcommand{\thepage} {\arabic{page}}

\section*{Research Objectives}
The over-arching goal of our project is to create a streamlined workflow
for experimental data analysis and interpretation
needs of the neutron scattering community using atomistic modeling and
simulation. Neutron scattering experiments require users to model and
interpret data at the atomic level. With numerous software
applications and a large array of different file formats with each,
scientists tend to use a limited (and sometimes dated) subset of
software tools to tackle data analysis from neutron experiments. This
creates a barrier in their research between experimental and theoretical techniques.

We are currently developing a modeling and analysis workbench called the 
Integrated Computational Environment-Modeling \& Analysis for Neutrons,
or ICE-MAN. This is an extension of the Eclipse ICE project \cite{ICEwebsite}. We hope to create a seamless transition from both different
types of neutron scattering experimental data and computer modeling and
simulation techniques to tackle multi-modal data
analysis. An example of a workflow would be the study of a disordered
material. First, neutron scattering experiments could yield the average
structure of the material (diffraction data) and
the local structure via total scattering (the pair
distribution function, or PDF).  Molecular dynamics (MD) simulations can provide
a trajectory for an atomistic model  giving an
ensemble of possible atomic configurations to compare to experiment.
Sampled configurations from the trajectory could be used as inputs into reverse Monte Carlo modeling (RMC) to optimize the structures against the the diffraction and PDF data produced from the neutron scattering experiments.
The MD trajectories could also then be used to calculate the
inelastic neutron scattering spectrum of the system. This data could
feed back into the RMC modeling as a constraint. At present this process
would be exceedingly time consuming, involve expertise in multiple
techniques, and most likely present a barrier that would
seldom be overcome by general users.

We are requesting XSEDE's HPC resources to carry out this workflow on the outlined projects that already have a need for computational approaches to solve complex structural data analysis and also multiple potential projects for the general neutron scattering community, specifically Users of the Spallation Neutron Source (SNS) instruments at Oak Ridge National Lab. We propose three projects that have already been studied via neutron scattering experiments and have a need for atomistic modeling to aid in the data analysis process. These include irradiated amorphous silica for nuclear radiation damage on materials, high-pressure amorphous silica for cheaper manufacturing of electronics, and oxygen insertion into shafarzikite-like structures for oxygen storage materials.


\subsection*{Proposed Research}
\subsubsection*{Project 1: Investigate structural modifications of irradiated silica for nuclear materials}\label{lang}

The first project would be the study of the local structure changes in amorphous silica (SiO2) due to radiation damage. Previous studies have looked at the fine structure of ion tracks (narrow trails of permanent damage along penetrating heavy ion pathways) in thin film amorphous silica using small angle x-ray scattering (SAXS) measurements combine with non-equilibrium MD modeling and simulation techniques. These studies revealed that these ion tracks formed a shell around the path of penetration through the sample which consisted of a core lower in density and a shell high in density. The non-equlibrium MD calculations were carried out using an ~30k atom system where the ion track was produced by instantaneous deposition of kinetic energy to the atoms in the simulation cell. More recently, simulation sizes of ~600k atoms have also been carried out to compare to SAXS measurements as well. We intend to carry out similar simulations of comparable size that can be used to elucidate the structural changes observed in neutron scattering experiments of the average and local structure of these polymorphs of silica. Atomistic configurations from the end of this trajectory can then be fed into the RMC modeling to optimize the structure against the experimental data. The results of this study can help understand the fundamental degradation of silica materials exposed radiation damage and also to future work to manipulate nanoclusters within solid silica materials. 

\subsubsection*{Project 2: Modeling of amorphous germanium and silicon to determine high-pressure phase transformations}\label{haberl}

The second project we would like to undertake is trying to determine different structural changes of high-pressure amorphous germanium (a-Ge) and silicon (a-Si) from different forms of compression or indentation loads that result in new materials for industry applications. Similar to our first project, we will be modeling systems based on silicon that have been probed via neutron scattering studies to determine structure changes while under high-pressure, being compressed in diamond anvil cells or being indented using diamond tips. Under compression, these samples become metallic and do not return to their initial state upon releasing the pressure. Based on either indentation or compression, different structures are observed with improved electronic and photovoltaic properties with cheaper manufacturing methods. 

Our modeling and simulation effort will attempt to find atomistic models that best fit neutron scattering experiments probing the average and local atomic structure via diffraction and pair distribution functions along with vibrational density of states from vibrational spectroscopy. To create models of a-Ge and a-Si in agreement with all experimental data is still a current research challenge, and to date no such model exists. We have developed a structural relaxation program based on the Wooten-Winer-Weaire method to produce realistic random-network models of a-Ge and a-Si via the Atomic Simulation Environment (ASE). From this program, we have initial amorphous structures that retain their four-fold coordination and bond-angle deviations in qualitative agreement with experiment. We use these structures to then perform a range of equilibrium or non-equilibrium MD simulations to best replicate different pressure loading and thermal annealing cycles. We feed atomic configurations from the MD trajectories into reverse RMC modeling to optimize the structure against the experimental data taken throughout thermal and pressure load cycles and the send this back to the MD simulations for further refinement and progess in the cycle. Experimentally constrained methods have been pursued in recent studies with success using very similar Monte Carlo relaxation methods. These include pure Monte Carlo using just the experimental data and hybrid approaches that add an emperical potential as an additional constraint. Our hope that using ICE-MAN's unique capability of lowering the barrier for data cross-over from the MD simulations to the RMC optimization to fit to experimental data will allow a larger search in the phase space and accelerate the convergence of a best fit to a variety of experimental data with strong emphasis on neutron data for these materials.
\subsubsection*{Project 3: Modeling oxygen insertion in one-dimensional channels of shafarzikite-like structures}\label{deLaune}
Support work in understanding the atomic-level structural changes in shafarzikite-like (FeSb2O4) structures. Will look specifically at the oxygen insertion into cobalt and lead doped derivatives of shafarzikite. These materials show promise in applications of electro-catalysis due to their unique 1-D cation channels with high peroxide anion mobility and potential for high, directed electronic conductivity. Will help with correction and analysis of the neutron scattering data to push modeling efforts to clarify the oxidized structures.


\section*{Computational Methodology (applications/codes)}
\subsection*{LAMMPS}
The main engine for the MD simulations will be the Large-Scale Atomistic/Molecular Massively Parallel Simulator (LAMMPS) software. This code is highly-scalable for a variety of machine architectures: IBM BG/L, Cray XT3, Cray XT5, and Intel clusters with GPUs (Comet). It can take advantage of GPU and Intel MIC accelerators with good performance scaling. For documentation on general benchmarking of LAMMPS, the following website contains several benchmark problems for a variety of machines: \href{http://lammps.sandia.gov/bench.html}{http://lammps.sandia.gov/bench.html}. 


\subsection*{DFT Codes}

Currently, the VISION neutron vibrationl spectrometer instrument at the SNS has its own dedicated computer cluster, VirtuES, for carrying out  computer modeling as integral part of the neutron data analysis and interpretation of the spectra (discussed more in the Additional Comments). This cluster has a variety of DFT codes installed for carrying out eletronic struturce calculations to determine vibrational density of states spectra. Vision Users have a suite of codes to choose from: VASP, Quantum ESPRESSO, CASTEP, ABINIT, CP2K, RMG-DFT, and GULP. Also, the O'Climax software is currently being developed by the VISION team as software that can generate calculated incoherent INS spectra from the output of these listed DFT codes. 

A major thrust of the ICE-MAN project is to have an interface to the O'Climax code within the first year of the project. This implies that ICE-MAN will also interface with these DFT codes to create the workflow between these codes and eventually feed in to the structural modeling. A subset of these codes will be used for our quantum calculation work so explicit scaling and performance measurements are not show. We will only present a small test case for the Quantum ESPRESSO code only for the schafarzikite project.

\subsection*{RMCProfile}
The majority of fitting to neutron scattering data will be carried out using the RMCProfile \cite{Tucker2007, RMCProfileWebsite} software to perform reverse Monte Carlo optimization modeling. This software is able to fit many data types simultaneously (Neutron and X-ray total scattering, Bragg diffraction profiles, EXAFS, and single crystal diffuse scattering) and use a range of constraints to produce atomic models that are consistent with all the available data. The fitting will be carried out in a "perfectly parallel" (or "embarrassingly parallel") manner where an array of fits will be carried out simultaneously on serial processors with no communication between the processes. This makes the Open Science Grid XSEDE resource optimal to carry out the fitting optimization. The ICE-MAN project is an extension of the ICE project \cite{ICEwebsite}, which is already able to handle the launching and monitoring of local and remote jobs, visualizing and analyzing data, and managing data transfers. Thus, we will work towards automating a seamless transition from taking data located on an HPC resource generated by atomistic modeling and simulation and transferring it to the High Throughput Computing (HTC) resource where it will automatically be launched for fitting optimization, preserving HPC service units for jobs that can make use of the resource. This project would provide an impetus to push parallelization of the RMCProfile software to handle the large atomistic configurations generated and make use of the HPC resources, reducing optimization timescales.



\subsection*{ICE-MAN}

Developing ICE-MAN on XSEDE resources would open up the machines that the software is available on and diversify the architecture development. Users at neutron scattering facilities who already have access to XSEDE resources or collaborate with other research teams that do could have ICE-MAN as a common platform to bring together their atomistic modeling data with their neutron experiment data.

\newpage

\subsection*{Comet}
Comet is the ideal resource for our research for the following reasons:

\begin{enumerate}

\item It provides a heterogeneous research platform that is capable of providing domain-specific, ideal architectures. The compute nodes are ideal for our large-scale atomistic simulations and the large memory nodes are ideal for our quantum calculation work. The GPU nodes are available to tackle very large problem sizes that make use of massive parallelization efficiently.

\item We are targeting the Oak Ridge National Laboratory Leadership Computing Facility resources (i.e. Titan) to launch ICE-MAN. However, due to all nodes containing a GPU on Titan, we can only use the machine to its full potential for a subset of our project. Using Comet as the first target machine allows us access to the resources we will eventually use, broaden the scope of machines that ICE-MAN can utilize, and be accessible to a larger part of the research community.

\item Members of the team already have experience using Comet and have had publications as a direct result of the allocations awarded on the machine.

\item The Data Oasis Lustre parallel file system ensures plenty of scalable storage available for the jobs run on the machine.

\end{enumerate}

\subsection*{Open Science Grid}
The Open Science Grid would be an optimal resource to use for launching multiple arrays of RMC jobs in a high throughput manner for large spanning of the phase space. The ICE-MAN project could take advantage on its already-underlying remote job launching capabilities to transfer work to the appropriate computational resource.


\subsection*{Performance and Scaling}
Classical/empirical potential MD simulations scale much better than their electronic structure counterparts due the parallel strategy being simpler and more straightforward. Thus, the larger jobs using the most amount of nodes in this project are the MD calculations. We focus below on the scaling and performance of the silica simulations using LAMMPS on Comet (CPU only simulations).

In Figures \ref{scaling_quartz} and \ref{scaling_twoTemp}, we show the benchmark calculations on Comet for both bulk quartz simulations and two-temperature cascade simulations, respectively. In these figures, we show the strong scaling (speedup and parallel efficiency for a fixed problem size), performance (MD steps per second) and weak scaling (fixed number of atoms per node) for each system. For the simulation setup, the system size consists of 648k atoms (Strong scaling: problem size, Weak scaling: atoms / node), we used a Tersoff empirical potential \cite{Tersoff1988, Kumagai2007}, and the simulations were for 100 MD steps with a 1 femtosecond timestep.

\begin{figure}[H]
  \begin{center}
 
  \begin{tabular}{cc}
    \multicolumn{2}{c}{\includegraphics[width=0.35\textwidth]{graphics/comet_strong_quartz.png}} \\
          \includegraphics[width=0.35\textwidth]{graphics/comet_perf_quartz.png} &
    \includegraphics[width=0.35\textwidth]{graphics/comet_weak_quartz.png} \\
  \end{tabular}
  \caption{\textbf{Performance and Scaling of LAMMPS for Quartz Simulations.} Simulation consists of 640k atoms, Tersoff Potential, $\Delta t$ = 1 fs, for 100 MD timesteps.}\label{scaling_quartz}
  
    \end{center}
\end{figure}

\begin{figure}[H]
  \begin{center}
      \begin{tabular}{cc}
    \multicolumn{2}{c}{\includegraphics[width=0.35\textwidth]{graphics/comet_strong_twoTemp.png}} \\
    \includegraphics[width=0.3\textwidth]{graphics/comet_perf_twoTemp.png} &
    \includegraphics[width=0.3\textwidth]{graphics/comet_weak_twoTemp.png} \\
    \end{tabular}
    
  \caption{\textbf{Performance and Scaling of LAMMPS for Irradiated Quartz via Two-Temperature Modeling Simulations.} Simulation consists of 640k atoms, Tersoff Potential, $\Delta t$ = 1 fs, for 100 MD timesteps.}\label{scaling_twoTemp}
  
  \end{center}
\end{figure}

\section*{Computational Research Plan}


\begin{description}
\item[Irradiated Amorphous Silica by Swift Heavy Ions] Swift heavy ion cascade simulations of amorphous silica will be carried out to produce atomistic configurations that can be directly compared and optimized to neutron scattering experiments via RMCProfile. This material will be first of many irradiated nuclear waste glass materials we will study via neutron scattering experiments in the upcoming year. ICE-MAN will be developed and optimized for the automated workflow between these modeling techniques: non-equilibrium MD simulations to RMC fitting of experimental data. Realistic modeling of the electron-phonon coupling via two-temperature modeling will also be explored. This will require electronic structure calculations to determine the temperature-dependent electronic specific heat, which will be an input into the SHI cascade simulations. The simulation trajectories will again be inputs into RMC modeling and fitted to neutron experimental data. 

\item[Amorphous Ge/Si via High-Pressure] a-Ge and a-Si will be modeled using varying compression and nanoindentation MD simulations to replicate experimental preparation methods. Using RMCProfile, atomic configurations from MD simulations will be used to optimize and fit against neutron scattering experiments. These optimized structures can be fed back into the MD simulations to overcome barriers not accessible in the temporal limits of MD or for virtual experiments to help direct future material processing methods. ICE-MAN will be developed to reduce the barrier in the forward and backward direction of data transfer between the MD and RMCProfile modeling.

\item[Oxygen Insertion in 1-D Cation Channel Materials]  RMCProfile will be used to clarify atomic structures of schafarzikite-like, 1-D cation channel materials before and after oxidation based on available neutron scattering data. The final structures can then be linked via electronic structure nudge elastic band calculations to determine minimum energy pathways for oxygen insertion into the lattice structure. ICE-MAN will again handle launching and data transfer workflow between the RMCProfile and quantum modeling calculations.



\end{description} 



\section*{Justification for Service Units (SUs) Requested}

For the irradiated amorphous silica by swift heavy ion project using the thermal spike via instantaneous deposition of kinetic energy, we plan to carry out upwards of 100 cascade simulations. We will use 8 nodes (192 processors) per simulation with a 1 femtosecond timestep with a total simulation time of 1 nanosecond (equilibration and production combined). Thus, given that it will take 0.009 seconds per MD step, we will require 48,000 service units. The two-temperature model for the electron-phonon coupling simulations will also be upwards of ~100 cascade simulations. We will use 5 nodes (120 processors) per simulation with the same timestep and simulation time. With 0.017 seconds per MD step, we will require 57,000 service units. Also, 
we request 50,000 service units for carrying out DFT calculations to determine electronic specific heat for input into the MD simulations. The Quantum ESPRESSO code is currently planned to be used for these calculations.  Thus, the total time will be 155,000 service units  for the silica project. However, we already have another proposed nuclear waste glass material that will be studied via neutron total scattering. This experiment could be completed as soon as April 2017. Thus, we request an additional 155,000 service units to also support this work, bringing the total requested service units to \textbf{310,000} for this part of the overall project. 

For the amorphous silica under high-pressure, we request 40,000 service units to carry out the parallel structure optimization jobs for producing the initial amorphous silicon and germanium configuration. We also plan to carry out upwards of 100 simulations under different pressure conditions (compression and nanoindentation). We will use 8 nodes per simulation with a 1 femtosecond timestep with a total simulation time of 20 nanosecond (cycling of equilibration and production for continual pressure loading). Thus, given that it will take 0.009 seconds per MD step, we will require 600,000 service units for this part of the project. This will bring the total requested service units to \textbf{640,000} for this part of the overall project. 

We request \textbf{100,000} service units for carrying out DFT calculations for the schafarzikite project. With poor scaling, we plan to use no more than 8 nodes per job. The Quantum ESPRESSO code is currently planned to be used for the schafarzikite calculations. These calculations will be shared between the VirtuES machine and Comet. 


We request \textbf{50,000} service units for running RMCProfile optimization modeling on the Open Science Grid. These calculations will be launched as arrays of serial jobs.

Table \ref{SU_table} summarizes the justification for the requested resources to begin our project.

\begin{table}[H]
  \caption{Summary of requested service units for projects}\label{SU_table}
  \resizebox{\textwidth}{!}{%
  \begin{tabular}{cccccc}
  	\toprule
  	Table & Project & Machine & Program & Service-Units & Storage (GB)\\
  	\midrule
  	1 & Irradiated SiO2   & Comet/Oasis & LAMMPS/QE & 310,00 & 500 \\
  	2 & High-pressure a-Ge/a-Si & Comet/Oasis & LAMMPS & 640,000 & 500\\
  	3 & Oxygen in FeSb2O4 & Comet/Oasis & QE & 100,000 & 100\\
  	4 & RMC modeling   & OSG & RMCProfile & 50,000 & 100 \\
  	\midrule
  	  & Total   & Comet &   & 1,100,000 &  \\
  	  &         & Oasis &   &           & 1,100 \\
  	  &         & OSG &     & 50,000    &  100 \\
  	\midrule
  	  & Grand Total   &     &           & \textbf{1,100,000} & \textbf{1,200}  \\
  	\bottomrule
  \end{tabular}
  }
\end{table}


\section*{Additional Comments}
VirtuES is available but is dedicated to VISION Users at the SNS. The machine consists of 50 nodes with each node consisting of two 16-core Intel Xeon E5-2698 v3 running at 2.30GHz.  

MTM has currently just finished using a initial startup allocation for 100,000 total SUs from a previous project on the two XSEDE HPC resources at the San Diego Supercomputer Center: Comet and Gordon (50,000 SUs each)


\bibliography{proposal}
\bibliographystyle{unsrt}


\end{document}

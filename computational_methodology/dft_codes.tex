Currently, the VISION neutron vibrationl spectrometer instrument at the SNS has its own dedicated computer cluster, VirtuES, for carrying out  computer modeling as integral part of the neutron data analysis and interpretation of the spectra (discussed more in the Additional Comments). This cluster has a variety of DFT codes installed for carrying out eletronic struturce calculations to determine vibrational density of states spectra. Vision Users have a suite of codes to choose from: VASP, Quantum ESPRESSO, CASTEP, ABINIT, CP2K, RMG-DFT, and GULP. Also, the O'Climax software is currently being developed by the VISION team as software that can generate calculated incoherent INS spectra from the output of these listed DFT codes. 

A major thrust of the ICE-MAN project is to have an interface to the O'Climax code within the first year of the project. This implies that ICE-MAN will also interface with these DFT codes to create the workflow between these codes and eventually feed in to the structural modeling. A subset of these codes will be used for our quantum calculation work so explicit scaling and performance measurements are not show. We will only present a small test case for the Quantum ESPRESSO code only for the schafarzikite project.
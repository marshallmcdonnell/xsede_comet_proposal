Currently, VISION \cite{VISIONwebsite} has its own dedicated computer cluster, Virtual Experiments  in Spectroscopy  with neutrons (VirtuES), for carrying out  computer modeling as an integral part of the neutron data analysis and interpretation of the spectra (discussed more in the Additional Comments). This cluster has a variety of DFT codes installed for carrying out electronic structure calculations to determine vibrational density of states spectra. VISION Users have a suite of codes to choose from: VASP \cite{Kresse1996, Kresse1996a, VASPwebsite}, Quantum ESPRESSO \cite{Giannozzi2009, QEwebsite}, CASTEP \cite{Clark2005, CASTEPwebiste} , ABINIT \cite{ Gonze2002, Gonze2009, CASTEPwebsite}, CP2K \cite{Hutter2014, CP2Kwebsite} , and RMG-DFT \cite{Briggs1996, RMGDFTwebsite}. Also, the O'Climax software ( the new version of aCLIMAX \cite{Ramirez-Cuest2004}) is currently being developed by the VISION team as software that can generate calculated incoherent INS spectra from the output of these listed DFT codes. 

A major thrust of the ICE-MAN project is to have an interface to the O'Climax code within the first year of the project. This implies that ICE-MAN will also interface with these DFT codes to create the workflow between these codes and eventually feed in to the structural modeling. 

The DFT jobs will be kept to a minimum on the Comet resource due to VirtuES being available for these calculations and scaling is harder to achieve with DFT codes compared to empirical potential calculations/simulations. Yet, the VirtuES resource is for Users of the VISION beamline and can become saturated with work outside the scope of the projects listed in this proposal. Comet would serve as a supplementary machine for majority of the quantum calculation work proposed here.  
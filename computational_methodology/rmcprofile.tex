The majority of fitting to neutron scattering data will be carried out using the RMCProfile \cite{Tucker2007, RMCProfileWebsite} software to perform reverse Monte Carlo optimization modeling. This software is able to fit many data types simultaneously (Neutron and X-ray total scattering, Bragg diffraction profiles, EXAFS, and single crystal diffuse scattering) and use a range of constraints to produce atomic models that are consistent with all the available data. The fitting will be carried out in a "perfectly parallel" (or "embarrassingly parallel") manner where an array of fits will be carried out simultaneously on serial processors with no communication between the processes. This makes the Open Science Grid XSEDE resource optimal to carry out the fitting optimization. The ICE-MAN project is an extension of the ICE project \cite{ICEwebsite}, which is already able to handle the launching and monitoring of local and remote jobs, visualizing and analyzing data, and managing data transfers. Thus, we will work towards automating a seamless transition from taking data located on an HPC resource generated by atomistic modeling and simulation and transferring it to the High Throughput Computing (HTC) resource where it will automatically be launched for fitting optimization, preserving HPC service units for jobs that can make use of the resource. This project would provide an impetus to push parallelization of the RMCProfile software to handle the large atomistic configurations generated and make use of the HPC resources, reducing optimization timescales.


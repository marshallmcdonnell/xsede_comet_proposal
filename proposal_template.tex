\documentclass{proposalnsf}
\usepackage{epsfig}

% NSF proposal generation template style file.
% based on latex stylefiles  written by Stefan Llewellyn Smith and
% Sarah Gille, with contributions from other collaborators.

\newcommand{\jas}{{\it J. Atmos. Sci.}}
\newcommand{\jpo}{{\it J. Phys. Oceanogr.}}
\newcommand{\JPO}{{\it J. Phys. Oceanogr.}}
\newcommand{\jfm}{{\it J. Fluid Mech.}}
\newcommand{\jgr}{{\it J. Geophys. Res.}}
\newcommand{\JGR}{{\it J. Geophys. Res.}}
\newcommand{\jmr}{{\it J. Mar. Res.}}
\newcommand{\arfm}{{\it Ann. Rev. Fluid Mech.}}
\newcommand{\dsr}{{\it Deep-Sea Res.}}
\newcommand{\dao}{{\it Dyn. Atmos. Oceans}}
\newcommand{\jam}{{\it Journal of Applied Meteorology}}
\newcommand{\phfl}{{\it Phys. Fluids}}
\newcommand{\phfla}{{\it Phys. Fluids A}}
\newcommand{\PhilTrans}{{\it Philosophical Transactions of the Royal Society, 
London}}
\newcommand{\gafd}{{\it Geophys. Astrophys. Fluid Dyn.}}
\newcommand{\gfd}{{\it Geophys. Fluid Dyn.}}
\newcommand{\PCE}   {{\it Physics and Chemistry of the Earth}}
\newcommand{\PRL}   {{\it Physical Review Letters}}

\newcommand{\ProgOc}{{\it Prog. Oceanography}}
\newcommand{\WHOITR}{Woods Hole Oceanographic Institution Technical Report, WHOI-}
\newcommand{\degrees}{$\!\!$\char23$\!$}
%%% old lines below defined some mathematical fonts; these no longer seem necessary
%\DeclareFontFamily{OT1}{psyr}{}
%\DeclareFontShape{OT1}{psyr}{m}{n}{<-> psyr}{}
%\def\times{{\fontfamily{psyr}\selectfont\char180}}


\renewcommand{\refname}{\centerline{References cited}}

% this handles hanging indents for publications
\def\rrr#1\\{\par
\medskip\hbox{\vbox{\parindent=2em\hsize=6.12in
\hangindent=4em\hangafter=1#1}}}

\def\baselinestretch{1}

\begin{document}


\noindent{\Large{\bf Computational Modeling to Aid in Analysis and Interpretation of Multi-Modal Neutron Experiments }}\\*[3mm]

\pagenumbering{arabic}
\renewcommand{\thepage} {\arabic{page}}



\section*{Research Objectives}

\subsection*{Significance of  }

The over-arching goal of our project is create a workflow for data analysis and interpretation
needs of the neutron scattering community through streamlined atomistic modeling. Neutron scattering experiments require users to model and interpret data at the atomic/molecular level. With numerous software applications and a large array of different file formats with each, scientists tend to use a limited (and sometimes dated) subset of software tools to tackle data analysis from neutron experiments. This creates a barrier to use other methods or atomistic modeling softwares in their research that could help in bridging the gap between experiment and theory. 

\subsection*{Proposed Research}

Propose the research projects that we can answer.  \\
  1) Maik's project \\
  2) Bianca's project \\
  3) Sankar's project \\
  4) Ben \& Colin's project \\


\section*{Computational Methodology (applications/codes)}

More text.

\section*{Computational Research Plan}

More text.  Cite an example \cite[]{sample_ref}

\section*{Justification for Service Units (SUs) Requested}

\section*{Additional Comments}





\newpage
\pagenumbering{arabic}
\renewcommand{\thepage} {E--\arabic{page}}

\bibliography{draft}
\bibliographystyle{jponew}

\newpage
\pagenumbering{arabic}
\renewcommand{\thepage} {G--\arabic{page}}
\noindent{\Large \bf BUDGET JUSTIFICATION}

\end{document}

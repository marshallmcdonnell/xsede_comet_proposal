\documentclass{proposalnsf}
\usepackage{epsfig}

% NSF proposal generation template style file.
% based on latex stylefiles  written by Stefan Llewellyn Smith and
% Sarah Gille, with contributions from other collaborators.

\newcommand{\jas}{{\it J. Atmos. Sci.}}
\newcommand{\jpo}{{\it J. Phys. Oceanogr.}}
\newcommand{\JPO}{{\it J. Phys. Oceanogr.}}
\newcommand{\jfm}{{\it J. Fluid Mech.}}
\newcommand{\jgr}{{\it J. Geophys. Res.}}
\newcommand{\JGR}{{\it J. Geophys. Res.}}
\newcommand{\jmr}{{\it J. Mar. Res.}}
\newcommand{\arfm}{{\it Ann. Rev. Fluid Mech.}}
\newcommand{\dsr}{{\it Deep-Sea Res.}}
\newcommand{\dao}{{\it Dyn. Atmos. Oceans}}
\newcommand{\jam}{{\it Journal of Applied Meteorology}}
\newcommand{\phfl}{{\it Phys. Fluids}}
\newcommand{\phfla}{{\it Phys. Fluids A}}
\newcommand{\PhilTrans}{{\it Philosophical Transactions of the Royal Society, 
London}}
\newcommand{\gafd}{{\it Geophys. Astrophys. Fluid Dyn.}}
\newcommand{\gfd}{{\it Geophys. Fluid Dyn.}}
\newcommand{\PCE}   {{\it Physics and Chemistry of the Earth}}
\newcommand{\PRL}   {{\it Physical Review Letters}}

\newcommand{\ProgOc}{{\it Prog. Oceanography}}
\newcommand{\WHOITR}{Woods Hole Oceanographic Institution Technical Report, WHOI-}
\newcommand{\degrees}{$\!\!$\char23$\!$}
%%% old lines below defined some mathematical fonts; these no longer seem necessary
%\DeclareFontFamily{OT1}{psyr}{}
%\DeclareFontShape{OT1}{psyr}{m}{n}{<-> psyr}{}
%\def\times{{\fontfamily{psyr}\selectfont\char180}}


\renewcommand{\refname}{\centerline{References cited}}

% this handles hanging indents for publications
\def\rrr#1\\{\par
\medskip\hbox{\vbox{\parindent=2em\hsize=6.12in
\hangindent=4em\hangafter=1#1}}}

\def\baselinestretch{1}

\begin{document}


\noindent{\Large{\bf Computational Modeling to Aid in Analysis and Interpretation of Multi-Modal Neutron Experiments }}\\*[3mm]

\pagenumbering{arabic}
\renewcommand{\thepage} {\arabic{page}}



\section*{Research Objectives}

\subsection*{Significance of  }

The over-arching goal of our project is create a streamlined workflow
for experimental data analysis and interpretation
needs of the neutron scattering community using atomistic modeling and
simulation. Neutron scattering experiments require users to model and
interpret data at the atomic/molecular level. With numerous software
applications and a large array of different file formats with each,
scientists tend to use a limited (and sometimes dated) subset of
software tools to tackle data analysis from neutron experiments. This
creates a barrier to use other methods or cutting-edge atomistic
modeling softwares in their research that could help in bridging the gap
between experiment and theory. 

We are currently developing a modeling and analysis workbench called
Integrated Computational Environment-Modeling \& Analysis for Neutrons,
or ICE-MAN. We hope to create a seamless transition from both different
types of neutron scattering experimental data and computer modeling and
simulation techniques to tackle multi-modal data
analysis. An example of a workflow would be the study of a disordered
material. First, neutron scattering experiments could yeild the average
structure of the material, providing Bragg diffraction data, and
the local structure via total scattering, providing the pair
distribution function.  Molecular dynamics (MD) simulations can provide
a trajectory for an atomistic model of the material in time giving an
ensemble of possible atomic configurations to compare to experiment.
These sampled
configurations from the trajectory could be converted and refined with
reverse Monte Carlo modeling (RMC) to determine which atomic configurations
produce the best fit to both the Bragg diffraction pattern and pair
distribution functions produced from the neutron scattering experiments.
The MD trajectories could also then feed into software to calculate the
inelastic neutron scattering spectrum of the system. This data could
feed back into the RMC modeling as a constraint. At present this process
would be exceedingly time consuming, involve expertise in at multiple
techniques, and most likely present a barrier that would
seldom be overcome by general users. 

We are requesting XSEDE's HPC resources to carry out this workflow on projects that are already ready to be fed into the pipeline and also for potential projects for the general neutron scattering community, specifically Users of the Spallation Neutron Source instruments.

\subsection*{Proposed Research}

Propose the research projects that we can answer.  \\
  1) Maik's project \\
  2) Bianca's project \\
  3) Sankar's project \\
  4) Ben \& Colin's project \\

\subsubsection*{Project 1: Investigate structural modifications of irradiated SiO2 for nuclear materials}

The first project would be the study of the local structure changes in amorphous silica (SiO2) due to radiation damage. Previous studies have looked at the fine structure of ion tracks (narrow trails of permanent damage along penetrating heavy ion pathways) in thin film amorphous silica using small angle x-ray scattering (SAXS) measurements combine with non-equilibrium MD modeling and simulation techniques. These studies revealed that these ion tracks formed a shell around the path of penetration through the sample which consisted of a core lower in density and a shell high in density. The non-equlibrium MD calculations were carried out using an ~30k atom system where the ion track was produced by instantaneous deposition of kinetic energy to the atoms in the simulation cell. More recently, simulation sizes of ~600k atoms have also been carried out to compare to SAXS measurements as well. We intend to carry out similar simulations of comparable size that can be used to elucidate the structural changes observed in neutron scattering experiments of the average and local structure of these polymorphs of silica. Atomistic configurations from the end of this trajectory can then be fed into the RMC modeling to optimize the structure against the experimental data. The results of this study can help understand the fundamental degradation of silica materials exposed radiation damage and also to future work to manipulate nanoclusters within solid silica materials. 




\section*{Computational Methodology (applications/codes)}

More text.

\section*{Computational Research Plan}

More text.  Cite an example \cite[]{sample_ref}

\section*{Justification for Service Units (SUs) Requested}

\section*{Additional Comments}





\newpage
\pagenumbering{arabic}
\renewcommand{\thepage} {E--\arabic{page}}

\bibliography{draft}
\bibliographystyle{jponew}

\newpage
\pagenumbering{arabic}
\renewcommand{\thepage} {G--\arabic{page}}
\noindent{\Large \bf BUDGET JUSTIFICATION}

\end{document}

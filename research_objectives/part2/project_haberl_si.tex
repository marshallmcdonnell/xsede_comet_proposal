\subsubsection*{Project 2: Modeling of amorphous germanium and silicon to determine high-pressure phase transformations}\label{haberl}

The second project we would like to undertake is trying to determine different structural changes of high-pressure amorphous germanium (a-Ge) and silicon (a-Si) from different forms of compression or indentation loads that result in new materials for industry applications. Similar to our first project, we will be modeling systems based on silicon that have been probed via neutron scattering studies to determine structure changes while under high-pressure, being compressed in diamond anvil cells or being indented using diamond tips. Under compression, these samples become metallic and do not return to their initial state upon releasing the pressure. Based on either indentation or compression, different structures are observed with improved electronic and photovoltaic properties with cheaper manufacturing methods. 

Our modeling and simulation effort will attempt to find atomistic models that best fit neutron scattering experiments probing the average and local atomic structure via diffraction and pair distribution functions along with vibrational density of states from vibrational spectroscopy. To create models of a-Ge and a-Si in agreement with all experimental data is still a current research challenge, and to date no such model exists. We have developed a structural relaxation program based on the Wooten-Winer-Weaire method to produce realistic random-network models of a-Ge and a-Si via the Atomic Simulation Environment (ASE). From this program, we have initial amorphous structures that retain their four-fold coordination and bond-angle deviations in qualitative agreement with experiment. We use these structures to then perform a range of equilibrium or non-equilibrium MD simulations to best replicate different pressure loading and thermal annealing cycles. We feed atomic configurations from the MD trajectories into reverse RMC modeling to optimize the structure against the experimental data taken throughout thermal and pressure load cycles and the send this back to the MD simulations for further refinement and progess in the cycle. Experimentally constrained methods have been pursued in recent studies with success using very similar Monte Carlo relaxation methods. These include pure Monte Carlo using just the experimental data and hybrid approaches that add an emperical potential as an additional constraint. Our hope that using ICE-MAN's unique capability of lowering the barrier for data cross-over from the MD simulations to the RMC optimization to fit to experimental data will allow a larger search in the phase space and accelerate the convergence of a best fit to a variety of experimental data with strong emphasis on neutron data for these materials.
\subsubsection*{Project 2: Modeling of amorphous germanium and silicon to determine high-pressure phase transformations}\label{haberl}

The second project is determining different structural changes of high-pressure amorphous germanium (a-Ge) and silicon (a-Si) from different forms of compression or indentation loads that result in new materials for industry applications \cite{Haberl2009, Borisenko2012, Haberl2014}. Similar to our first project, we will be modeling systems based on silicon that have been probed via neutron scattering carried out under high-pressure, being compressed in diamond anvil cells or being indented using diamond tips. These experiments were carried out at the Spallation Neutron and Pressure Diffractometer (SNAP) \cite{SNAPwebsite} at the SNS. Upon releasing the pressure load, these samples do not return to their initial state but stay in a different, metastable state. Based on either indentation or compression, improved electronic and photovoltaic properties have been observed in these materials. Also, high-pressure processing methods from this study could provide cheaper manufacturing methods for electronics \cite{Holmstrom2016}.

We will determine atomistic models that best fit neutron scattering experiments probing the average and local atomic structure via diffraction and pair distribution functions from experiments on both SNAP and NOMAD, along with vibrational density of states from vibrational spectroscopy carried out on the VISION neutron vibrational spectrometer instrument \cite{VISIONwebsite} at the SNS. To create models of a-Ge and a-Si in agreement with a large range of experimental data is still a current research challenge, and to date no such model exists. We have developed a structural relaxation program based on the Wooten-Winer-Weaire (WWW) method \cite{Wooten1985} to produce amorphized structures of a-Ge and a-Si via the Atomic Simulation Environment (ASE) \cite{ Bahn2002, ASEwebsite} from initially crystalline structures. Using this technique, we get amorphous structures that retain their four-fold coordination and bond-angle deviations in qualitative agreement with experiments\cite{Wooten1985}.  Equilibrium MD simulations can be used to further relax structures and non-equilibrium MD simulations can be used to replicate different experimental pressure loading and thermal annealing cycles. Atomic configurations from the MD trajectories are inputs into reverse RMC modeling to optimize the structure against the experimental data. Our hope that using ICE-MAN's unique capability of lowering the barrier for data cross-over from the MD simulations to the RMC optimization to fit to experimental data will allow a larger and faster search in the phase space and accelerate the convergence of a best fit to a variety of experimental data with strong emphasis on neutron data for these materials.
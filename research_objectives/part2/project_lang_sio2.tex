\subsubsection*{Project 1: Investigate structural modifications of irradiated silica for nuclear materials}\label{lang}

The first project would be the study of the local structure changes in amorphous silica (SiO2) due to radiation damage. Previous studies have looked at the fine structure of ion tracks (narrow trails of permanent damage along penetrating heavy ion pathways) in thin film amorphous silica using small angle x-ray scattering (SAXS) measurements combine with non-equilibrium MD modeling and simulation techniques. These studies revealed that these ion tracks formed a shell around the path of penetration through the sample which consisted of a core lower in density and a shell high in density. The non-equlibrium MD calculations were carried out using an ~30k atom system where the ion track was produced by instantaneous deposition of kinetic energy to the atoms in the simulation cell. More recently, simulation sizes of ~600k atoms have also been carried out to compare to SAXS measurements as well. We intend to carry out similar simulations of comparable size that can be used to elucidate the structural changes observed in neutron scattering experiments of the average and local structure of these polymorphs of silica. Atomistic configurations from the end of this trajectory can then be fed into the RMC modeling to optimize the structure against the experimental data. The results of this study can help understand the fundamental degradation of silica materials exposed radiation damage and also to future work to manipulate nanoclusters within solid silica materials. 

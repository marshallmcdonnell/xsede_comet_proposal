\subsection*{Significance of  }

The over-arching goal of our project is create a streamlined workflow
for experimental data analysis and interpretation
needs of the neutron scattering community using atomistic modeling and
simulation. Neutron scattering experiments require users to model and
interpret data at the atomic/molecular level. With numerous software
applications and a large array of different file formats with each,
scientists tend to use a limited (and sometimes dated) subset of
software tools to tackle data analysis from neutron experiments. This
creates a barrier to use other methods or cutting-edge atomistic
modeling softwares in their research that could help in bridging the gap
between experiment and theory.

We are currently developing a modeling and analysis workbench called
Integrated Computational Environment-Modeling \& Analysis for Neutrons,
or ICE-MAN. We hope to create a seamless transition from both different
types of neutron scattering experimental data and computer modeling and
simulation techniques to tackle multi-modal data
analysis. An example of a workflow would be the study of a disordered
material. First, neutron scattering experiments could yeild the average
structure of the material, providing Bragg diffraction data, and
the local structure via total scattering, providing the pair
distribution function.  Molecular dynamics (MD) simulations can provide
a trajectory for an atomistic model of the material in time giving an
ensemble of possible atomic configurations to compare to experiment.
These sampled
configurations from the trajectory could be converted and refined with
reverse Monte Carlo modeling (RMC) to determine which atomic configurations
produce the best fit to both the Bragg diffraction pattern and pair
distribution functions produced from the neutron scattering experiments.
The MD trajectories could also then feed into software to calculate the
inelastic neutron scattering spectrum of the system. This data could
feed back into the RMC modeling as a constraint. At present this process
would be exceedingly time consuming, involve expertise in at multiple
techniques, and most likely present a barrier that would
seldom be overcome by general users.

We are requesting XSEDE's HPC resources to carry out this workflow on projects that are already ready to be fed into the pipeline and also for potential projects for the general neutron scattering community, specifically Users of the Spallation Neutron Source instruments.

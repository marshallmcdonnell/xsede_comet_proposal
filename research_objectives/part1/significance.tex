The over-arching goal of our project is to create a streamlined workflow
for experimental data analysis and interpretation
needs of the neutron scattering community using atomistic modeling and
simulation. Neutron scattering experiments require users to model and
interpret data at the atomic level. With numerous software
applications and a large array of different file formats with each,
scientists tend to use a limited (and sometimes dated) subset of
software tools to tackle data analysis from neutron experiments. This
creates a barrier in their research between experimental and theoretical techniques.

We are currently developing a modeling and analysis workbench called the 
Integrated Computational Environment-Modeling \& Analysis for Neutrons,
or ICE-MAN. This is an extension of the Eclipse ICE project \cite{ICEwebsite}. We hope to create a seamless transition from both different
types of neutron scattering experimental data and computer modeling and
simulation techniques to tackle multi-modal data
analysis. An example of a workflow would be the study of a disordered
material. First, neutron scattering experiments could yield the average
structure of the material (diffraction data) and
the local structure via total scattering (the pair
distribution function, or PDF).  Molecular dynamics (MD) simulations can provide
a trajectory for an atomistic model  giving an
ensemble of possible atomic configurations to compare to experiment.
Sampled configurations from the trajectory could be used as inputs into reverse Monte Carlo modeling (RMC) to optimize the structures against the the diffraction and PDF data produced from the neutron scattering experiments.
The MD trajectories could also then be used to calculate the
inelastic neutron scattering spectrum of the system. This data could
feed back into the RMC modeling as a constraint. At present this process
would be exceedingly time consuming, involve expertise in multiple
techniques, and most likely present a barrier that would
seldom be overcome by general users.

We are requesting XSEDE's HPC resources to carry out this workflow on the outlined projects that already have a need for computational approaches to solve complex structural data analysis and also multiple potential projects for the general neutron scattering community, specifically Users of the Spallation Neutron Source (SNS) instruments at Oak Ridge National Lab. We propose three projects that have already been studied via neutron scattering experiments and have a need for atomistic modeling to aid in the data analysis process. These include irradiated amorphous silica for nuclear radiation damage on materials, high-pressure amorphous silica for cheaper manufacturing of electronics, and oxygen insertion into shafarzikite-like structures for oxygen storage materials.
